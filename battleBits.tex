\documentclass[a4paper, 12pt]{article}

\usepackage[swedish]{babel}

\title{Sänka Micro:bits}
\author{Alex, David, Hugo, Jon, Momo, William \\ kandDV1.B2}
\pagenumbering{Roman}

\begin{document}
    \maketitle
    \tableofcontents
    \newpage
    \section{Bakgrund}
    \subsection*{20 augusti 2148}
    Jag vaknar och tittar ut genom fönstret. Det är svart. 
    Kylan tränger sig igenom metallväggarna och mullret från motorerna ekar i väggarna.
    Jag tar på mig min gråa uniform och går upp till den stora matsalen.
    När jag sätter mig ner för att äta märker jag hur tyst det är, ingen pratar.
    Tallriken gör inte det hela bättre. Allt som ligger där är bara brunt.
    Jag äter klart och går tillbaka till mitt rum för att borsta mina tänder.
    Då ljuder larmet, någon har skjutit. Vi måste flytta på oss.

    \subsection*{19 augusti 2148}
    Jag vaknar, solen skiner och det är varmt.
    Efter att jag tagit mig upp går jag till frukostbordet. Idag är det fyllt med frukt och bär att ha i yogurten.
    När jag gjort mig färdig är det dags att åka till jobbet. Jag sätter mig i bilen och börjar åka.
    Väl framme sätter jag på mig min uniform, går på toaletten och sätter mig på mitt första möte.
    Mitt i mötet hör vi det, larmet. Vi är i krig.
    Alla skyndar sig till sina poster, några går till helikoptrar, några till bunkrar och andra till båtar.
    Jag och mina kollegor går till en ubåt. Ingen säger något, alla är rädda.

    \newpage
    \section{Förberedelser}
    \subsection{Välja antalet spelare}
    Alla startar sina Micro:bits.
    En person väljer hur många som ska spela genom att 
    trycka på knapparna a och b och sedan skaka för att 
    låsa in siffran. \\

    \subsubsection{Kontroller för att välja antalet spelare}
    \noindent
    Knapp A\@: Minska siffran med 1 \\
    Knapp B\@: Öka siffran med 1 \\
    Skaka: Lås in siffran \\

    \subsection{Välja position}
    \noindent
    När en person har valt antalet spelare får alla spelare en prick i mitten av deras micro:bits.
    Detta är deras skepp. För att flytta skeppet,
    luta micro:biten åt det håll du vill att skeppet ska flyttas.

    \subsubsection{Kontroller för att välja position på skeppet}\label{sec:positions-kontroller}
    \noindent
    Luta vänster: Flytta åt vänster \\
    Luta höger: Flytta åt höger \\
    Luta framåt: Flytta uppåt \\
    Luta bakåt: Flytta nedåt \\
    Tryck på knapp A och knapp B\@: Lås in position \\

    \newpage
    \section{Spelets gång}
    Varje runda börjar med att alla spelare väljer var de vill skjuta samtidigt.
    Detta görs på samma sätt som att välja ditt skepps position (se~\ref{sec:positions-kontroller}).
    När alla spelare valt var de vill skjuta blir det missilregn.
    Alla spelare skjuter på alla.
    Du ser var alla andra har skjutit genom att missilerna visas som blinkande rutor.
    Om du har blivit träffad är du ute, annars så lever du vidare till nästa runda där du får skjuta igen.
    Den som lever sist vinner.
\end{document}